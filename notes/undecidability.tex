\section{Countability}

\vspace{\parskip}

\begin{definition}[Countable Set]
    A set $A$ is countable if $A$ is empty or there is a injective function $f:\, A \to \N$, or equivalently, a surjective function $f:\, \N \to A$.
\end{definition}

\begin{theorem}
    $\R$ is not countable.
\end{theorem}

\begin{theorem}
    $2^{\N} = \mathcal{P}(\N)$ is not countable.
\end{theorem}

\begin{corollary}
    There exists some $A \subseteq \Sigma^*$ such that $A$ is not Turing-recognizable.
\end{corollary}

\begin{proof}
    We will prove this by showing that the set of all languages in $\Sigma^*$ is not countable, but the set of Turing-recognizable languages is countable.

    We first show that $\Sigma^*$ is countable. To show this, we simply list all strings in $\Sigma^*$ in lexicographic order.

    Observe that $\mathsf{TR} = \{ A \subseteq \Sigma^* \mid A = L(M) \}$ is also countable because the number of Turing machines is countable, each of which can be encoded as a finite length string.

    But the set of all languages $\{ A \subseteq \Sigma^* \} = 2^{\Sigma^*}$. By Cantor's theorem, since $\Sigma^*$ is countably infinite, the power set of $\Sigma^*$ is uncountable.

    Hence, there exists some language $A \subseteq \Sigma^*$ that is not Turing-recognizable.
\end{proof}

Let $\encoding{M}$ denote the encoding of a Turing machine $M$. For $n \in \N$, let $\langle n \rangle \in \{0,1\}^*$ be the binary encoding of $n$. For a sequence of numbers $n_1,n_2,\ldots,n_k$, $\langle n_1,n_2,\ldots,n_k \rangle \in \{0,1,\#\}^*$ be $\langle n_1 \rangle \# \langle n_2 \rangle \# \cdots \# \langle n_k \rangle$.

Then, any Turing machine can be determined by a finite sequence of numbers specifying $|Q|$, $|\Sigma|$, $|\Gamma|$, identity of $q_0,q_{accept},q_{reject}$, and a sequence of numbers specifying $\delta$.

$$
\encoding{M} = \text{the string that determines $M$}
$$

Furthermore, we can denote a Turing machine $T$ with input $w$ as $\langle M, w \rangle$.

\begin{theorem}
    Let
    $$
    \mathsf{DIAG} = \{ \encoding{M} \mid \text{$M$ is a TM and $\encoding{M} \not\in L(M)$} \}
    $$
    $\mathsf{DIAG}$ is not Turing recognizable.
\end{theorem}

\begin{proof}
    Suppose, for contradiction, that there exists a TM $M$ such that $T(M) = \mathsf{DIAG}$. Then, we can ask if $\encoding{M} \in L(M)$.

    Suppose that $\encoding{M} \in L(M)$. Then
    $$
    \encoding{M} \in L(M) \iff \encoding{M} \in \mathsf{DIAG} \iff \encoding{M} \not\in \in L(M)
    $$
    which is a contradiction.
\end{proof}

This is similar to Russell's paradox.

\begin{theorem}
    Every finite language is Turing recognizable and decidable.
\end{theorem}

\begin{theorem}
    Every regular language is Turing recognizable and decidable.
\end{theorem}

\section{Universal Turing Machine}

Let $A_{\mathsf{TM}}$ be the language
$$
\{ \langle M,w \rangle \mid \text{$M$ accepts $w$} \}
$$

\begin{theorem}
    $A_{\mathsf{TM}}$ is Turing-recognizable.
\end{theorem}

\begin{proof}
    We show that there exists a TM $U$ such that $L(U) = A_{\mathsf{TM}}$. $U$ decodes $\encoding{M}$ into a TM $M$ and simulates $M$ on $w$. Thus, $A_{\mathsf{TM}}$ is Turing-recognizable.
\end{proof}

The $U$ that we have constructed to recognize $A_{\mathsf{TM}}$ is called a universal Turing machine.

\begin{theorem}
    $A_{\mathsf{TM}}$ is not decidable.
\end{theorem}

\section{Decidability}

\begin{theorem}
    A language $L$ is decidable if and only if there is an enumerator that enumerates $L$ in standard string order.
\end{theorem}

\begin{proof}
    Decidable implies standard string order enumerator.

    Let $M$ be a decider for $L$. We will design an standard string order.

    \begin{codebox}
        \Procname{$\proc{E}$}
        \li \For $w$ in standard string order \Do
            \li \textbf{run} $M$ \textbf{on} $w$ 
            \li \If $M$ accepts $w$ \Then
                \li print($w$)
    \end{codebox}

    Standard string enumerator implies decidable.

    \begin{codebox}
        \Procname{$\proc{M}(w)$}
        \li \textbf{run} $E$ 
        \li \While $\proc{Has-Next}(E)$ \Do \RComment{while $E$ still has string to print}
            \li $x = \proc{Next}(E)$ \RComment{get next string}
            \li \If $x \isequal w$ \Then
                \li \textbf{accept}
            \End
            \li \If $x \geq w$ \Then
                \li \textbf{reject}
    \end{codebox}
\end{proof}